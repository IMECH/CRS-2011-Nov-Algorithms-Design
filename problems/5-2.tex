\begin{problem}[习题5.2]
(双机调度问题)用两台处理机A和B处理个作业. 设第$i$个作业交给机器A处理时所需要的时间是$a_i$,
若由机器B来处理, 则所需要的时间是$b_i$. 现在要求每个作业只能由一台机器处理, 每台机器都不能
同时处理两个作业. 设计一个动态规划算法, 使得这两台机器处理完这个作业的时间最短(从任何一台机器开工到最后一台机器停工的总的时间). 以下面的例子说明你的算法:
\[
n = 6, {~} (a_1, a_2, a_3, a_4, a_5, a_6) = (2, 5, 7, 10, 5, 2), {~} (b_1, b_2, b_3, b_4, b_5, b_3) = (3, 8, 4, 11, 3, 4)
\]
\end{problem}
\begin{solution}
\textbf{解:}设给定的作业标号为$S={1,2,\cdots,n}$, T(k)为前k个作业用分给A,B机处理所需最小时间, 其中用A机处理的作业标号集合为I, 用B机处理的作业标号集合为J. 显然有$I + J = S$. 则有
\[
T(k) = \max\{\sum\limits_{i\in I}a_i, \sum\limits_{j\in J}b_j\}
\]


\[
T(k+1) = \min
\Big\{
         \textcolor{blue}{\max\{
              a_{k+1}+\sum\limits_{i\in I}a_i, \sum\limits_{j\in J}b_j
             \}}, {~}
          \textcolor{red}{\max\{
               \sum\limits_{i\in I}a_i, b_{k+1}+\sum\limits_{j\in J}b_j
               \}}
\Big\}
\]
因此$T(k)$具有最优子结构. 由以上分析可得出双机调度问题的动态规划算法, 其伪代码见表\ref{Schedule2Machine}.
\newpage
\begin{table}[!htb]
\centering
\caption{\label{Schedule2Machine}双机调度问题的动态规划算法伪代码}
\begin{tabular}{llll}
\hline
\multicolumn{4}{l}{\textbf{Proc} Schedule2Machine(a, b)} \\
 & \multicolumn{3}{l}{Ta = 0; Tb = 0;} \\
 & \multicolumn{3}{l}{n = $|a|$;} \\
 & \multicolumn{3}{l}{Machine(1:n) = 0; \textcolor{blue}{//标记作业$i$由第Machine(i)个机器完成}} \\
 & \multicolumn{3}{l}{\textbf{for} i \textbf{from} 1 \textbf{to} n} \\
 &  & \multicolumn{2}{l}{$\mathrm{Ta_i}$ = max\{Ta+a[i], Tb\};} \\
 &  & \multicolumn{2}{l}{$\mathrm{Tb_i}$ = max\{Tb+b[i],Ta\};} \\
 &  & \multicolumn{2}{l}{\textbf{if} $\mathrm{Ta_i < Tb_i}$ \textbf{then}} \\
 &  &  & Ta = Ta + a[i]; \\
 &  &  & Machine(i) = 1; \\
 &  & \multicolumn{2}{l}{\textbf{else}} \\
 &  &  & Tb = Tb + b[i]; \\
 &  &  & Machine(i) = 2; \\
 &  & \multicolumn{2}{l}{\textbf{end\{if\}}} \\
 & \multicolumn{3}{l}{\textbf{end\{for\}}} \\
 & \multicolumn{3}{l}{Tmin = \textbf{max}\{Ta, Tb\};} \\
 & \multicolumn{3}{l}{\textbf{return} Tmin, Machine;} \\
\multicolumn{4}{l}{\textbf{end\{Schedule2Machine\}}} \\
\hline
\end{tabular}
\end{table}
现以$n = 6$, $a[1:6] = (2, 5, 7, 10, 5, 2)$, $b[1:6] = (3, 8, 4, 11, 3, 4)$为例说明该算法执行的每一步,其中每步循环后, 程序中的各变量值见表\ref{variable}.
\begin{table}[!htb]
\centering
\caption{\label{variable}每步循环后的各变量值}
\begin{tabular}{|l|lll|l|l|l|l|}
\hline
\multicolumn{1}{|c|}{i} & \multicolumn{1}{c}{$\mathrm{Ta_i}$} & \multicolumn{1}{c}{} & \multicolumn{1}{c|}{$\mathrm{Tb_i}$} & \multicolumn{1}{c|}{Ta} & \multicolumn{1}{c|}{Tb} & \multicolumn{1}{c|}{T=max\{Ta,Tb\}} & \multicolumn{1}{c|}{Machine} \\
\hline
\multicolumn{1}{|c|}{0} & \multicolumn{1}{c}{0} & \multicolumn{1}{c}{} & \multicolumn{1}{c|}{0} & \multicolumn{1}{c|}{0} & \multicolumn{1}{c|}{0} & \multicolumn{1}{c|}{0} & \multicolumn{1}{c|}{[0,0,0,0,0,0]} \\
\hline
\multicolumn{1}{|c|}{1} & \multicolumn{1}{c}{2} & \multicolumn{1}{c}{$<$} & \multicolumn{1}{c|}{3} & \multicolumn{1}{c|}{3} & \multicolumn{1}{c|}{0} & \multicolumn{1}{c|}{3} & \multicolumn{1}{c|}{[\textbf{1},0,0,0,0,0]} \\
\cline{1-1}\cline{5-8}
\multicolumn{1}{|c|}{2} & \multicolumn{1}{c}{7} & \multicolumn{1}{c}{$<$} & \multicolumn{1}{c|}{8} & \multicolumn{1}{c|}{7} & \multicolumn{1}{c|}{0} & \multicolumn{1}{c|}{7} & \multicolumn{1}{c|}{[1,\textbf{1},0,0,0,0]} \\
\cline{1-1}\cline{5-8}
\multicolumn{1}{|c|}{3} & \multicolumn{1}{c}{14} & \multicolumn{1}{c}{$>$} & \multicolumn{1}{c|}{7} & \multicolumn{1}{c|}{7} & \multicolumn{1}{c|}{4} & \multicolumn{1}{c|}{7} & \multicolumn{1}{c|}{[1,1,\textbf{2},0,0,0]} \\
\cline{1-1}\cline{5-8}
\multicolumn{1}{|c|}{4} & \multicolumn{1}{c}{17} & \multicolumn{1}{c}{$>$} & \multicolumn{1}{c|}{15} & \multicolumn{1}{c|}{7} & \multicolumn{1}{c|}{15} & \multicolumn{1}{c|}{15} & \multicolumn{1}{c|}{[1,1,2,\textbf{2},0,0]} \\
\cline{1-1}\cline{5-8}
\multicolumn{1}{|c|}{5} & \multicolumn{1}{c}{15} & \multicolumn{1}{c}{$<$} & \multicolumn{1}{c|}{18} & \multicolumn{1}{c|}{12} & \multicolumn{1}{c|}{15} & \multicolumn{1}{c|}{15} & \multicolumn{1}{c|}{[1,1,2,2,\textbf{1},0]} \\
\cline{1-1}\cline{5-8}
\multicolumn{1}{|c|}{6} & \multicolumn{1}{c}{15} & \multicolumn{1}{c}{$<$} & \multicolumn{1}{c|}{19} & \multicolumn{1}{c|}{14} & \multicolumn{1}{c|}{15} & \multicolumn{1}{c|}{15} & \multicolumn{1}{c|}{[1,1,2,2,1,\textbf{1}]} \\
\hline
\end{tabular}
\end{table}

最终可以算得最小时间$T_{\min} = 15$; 各作业所有的机器$\textrm{Machine}=[1,1,2,2,1,1]$, $\textrm{Machine}(i)= 1$表示第$i$个作业由A机完成, $\textrm{Machine}(i)= 2$表示第$i$个作业由B机完成.

\end{solution}
