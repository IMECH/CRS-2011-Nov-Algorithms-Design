\begin{problem}[习题2.1]
证明下列结论:
\begin{enumerate}
\item 在一个无向图中, 如果每个顶点的度大于等于2, 则该该图一定含有圈;
\item 在一个有向图D中, 如果每个顶点的出度都大于等于1, 则该图一定含有一个有向圈.
\end{enumerate}
\end{problem}
\begin{solution}
\begin{enumerate}
\item \textbf{证:} 设该无向图无圈,有$m$条边,$n$个顶点. 则该无向图为树或森林,因此满足$m=n-k$,又由如果每个顶点的度大于等于2有
\[
\sum d(v) = 2|E| = 2m \geq 2n
\]
故有
\begin{equation}\label{q1}
m \geq n \Longrightarrow n - k \geq n, (k = 1,2,\cdots)
\end{equation}
显然式(\ref{q1})是不成立的,故假设该无向图无圈成立,即该该图一定含有圈.
\item \textbf{证:} 设该有向图中的最长有向迹为$P=V_1V_2\cdots V_k$,
又因为每个顶点的度大于等于1, 故存在$V'$为$V_k$连接到的点,若$V' \notin P$, 则$V_1V_2\cdots V_kV'$比$P$更长,与假设矛盾. 因此, $V'\in P$, 设$V'=V_n(n\leq k)$, 故存在有向圈$V_nV_{n+1}\cdots V_kV_n$
\end{enumerate}
\end{solution}
