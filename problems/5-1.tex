\begin{problem}[习题5.1]
最大子段和问题:给定整数序列$a_1, a_2, \cdots, a_n$, 求该序列形如$\sum_{k=i}^j a_k$的子段和的最大值:
$\max\{0, \max\limits_{1\leq i\leq j\leq n}\sum_{k=i}^j a_k\}$.
\begin{enumerate}
\item 已知一个简单算法如下:
\begin{lstlisting}[language=C++][caption={c++}]
int Maxsum(int n,int a,int& besti,int& bestj)
{ int sum = 0;
  for(int i=1;i<=n;i++){
     int suma = 0;
     for(int j=i;j<=n;j++){
        suma + = a[j];
        if(suma > sum){
          sum = suma;
          besti = i;
          bestj = j; }
     }
  }
  return sum;
}
\end{lstlisting}
试分析该算法的时间复杂性.
\item 试用分治算法解最大子段和问题, 并分析算法的时间复杂性.
\item 试说明最大子段和问题具有最优子结构性质, 并设计一个动态规划算法解最大子段和问题. 分析算法的时间复杂度.\\
(提示: 令$b(j) = \max\limits_{1\leq i\leq j\leq n}\sum_{k=i}^j a_k, j = 1, 2, \cdots, n$)
\end{enumerate}
\end{problem}
\begin{solution}
\begin{enumerate}
\item \textbf{解:}该简单算法的执行步数统见表\ref{stepnum}.
\begin{table}[!htb]
\centering
\caption{\label{stepnum}执行步数统计表}
\begin{tabular}{|l|l|l|l|l|l|l|l|l|}
\cline{1-4}\cline{6-9}
\multicolumn{1}{|c|}{行数} & \multicolumn{1}{c|}{s/e} & \multicolumn{1}{c|}{频率} & \multicolumn{1}{c|}{总步数} & \multicolumn{1}{c|}{} & \multicolumn{1}{c|}{行数} & \multicolumn{1}{c|}{s/e} & \multicolumn{1}{c|}{频率} & \multicolumn{1}{c|}{总步数} \\
\cline{1-4}\cline{6-9}
\multicolumn{1}{|c|}{01} & \multicolumn{1}{c|}{0} & \multicolumn{1}{c|}{0} & \multicolumn{1}{c|}{0} & \multicolumn{1}{c|}{} & \multicolumn{1}{c|}{08} & \multicolumn{1}{c|}{1} & \multicolumn{1}{c|}{n(n-1)/2} & \multicolumn{1}{c|}{n(n-1)/2} \\
\multicolumn{1}{|c|}{02} & \multicolumn{1}{c|}{0} & \multicolumn{1}{c|}{0} & \multicolumn{1}{c|}{0} & \multicolumn{1}{c|}{} & \multicolumn{1}{c|}{09} & \multicolumn{1}{c|}{1} & \multicolumn{1}{c|}{n(n-1)/2} & \multicolumn{1}{c|}{n(n-1)/2} \\
\multicolumn{1}{|c|}{03} & \multicolumn{1}{c|}{1} & \multicolumn{1}{c|}{n+1} & \multicolumn{1}{c|}{n+1} & \multicolumn{1}{c|}{} & \multicolumn{1}{c|}{10} & \multicolumn{1}{c|}{1} & \multicolumn{1}{c|}{n(n-1)/2} & \multicolumn{1}{c|}{n(n-1)/2} \\
\multicolumn{1}{|c|}{04} & \multicolumn{1}{c|}{1} & \multicolumn{1}{c|}{n+1} & \multicolumn{1}{c|}{n+1} & \multicolumn{1}{c|}{} & \multicolumn{1}{c|}{11} & \multicolumn{1}{c|}{0} & \multicolumn{1}{c|}{0} & \multicolumn{1}{c|}{0} \\
\multicolumn{1}{|c|}{05} & \multicolumn{1}{c|}{1} & \multicolumn{1}{c|}{n(n+1)/2} & \multicolumn{1}{c|}{n(n+1)/2} & \multicolumn{1}{c|}{} & \multicolumn{1}{c|}{12} & \multicolumn{1}{c|}{0} & \multicolumn{1}{c|}{0} & \multicolumn{1}{c|}{0} \\
\multicolumn{1}{|c|}{06} & \multicolumn{1}{c|}{1} & \multicolumn{1}{c|}{n(n-1)/2} & \multicolumn{1}{c|}{n(n-1)/2} & \multicolumn{1}{c|}{} & \multicolumn{1}{c|}{13} & \multicolumn{1}{c|}{0} & \multicolumn{1}{c|}{0} & \multicolumn{1}{c|}{0} \\
\multicolumn{1}{|c|}{07} & \multicolumn{1}{c|}{1} & \multicolumn{1}{c|}{n(n-1)/2} & \multicolumn{1}{c|}{n(n-1)/2} & \multicolumn{1}{c|}{} & \multicolumn{1}{c|}{14} & \multicolumn{1}{c|}{0} & \multicolumn{1}{c|}{0} & \multicolumn{1}{c|}{0} \\
\cline{1-4}\cline{6-9}
\end{tabular}
\end{table}
由表\ref{stepnum}可知该简单算法执行的总步数为:
$2\times (n+1)+n(n+1)/2 + 5\times n(n-1)/2 = 6n^2+4$
 所以该算法时间复杂度为$O(n^2)$.

\item 解最大子段的分治算法: 对给定序列$a[1:n]$对半划分为$a[1:\lfloor n/2\rfloor]$, $a[\lfloor n/2\rfloor+1:n]$, 并分别求出这两段的最大子段. 则原序列的最大子段有以下可能:
    \begin{itemize}
    \item 原序列的最大子段 = $a[1:\lfloor n/2\rfloor]$最大子段.
    \item 原序列的最大子段 = $a[\lfloor n/2\rfloor+1:n]$最大子段.
    \item 原序列的最大子段 = $a[1:\lfloor n/2\rfloor]$最大子段与$a[\lfloor n/2\rfloor+1:n]$最大子段组合.
    \end{itemize}
基于以上分析, 可得到解最大子段分治算法的伪代码, 见表\ref{MaxSubSum}.
\begin{table}[!htb]
\centering
\caption{\label{MaxSubSum}解最大子段分治算法的伪代码}
\begin{tabular}{lllll}
\hline
\multicolumn{5}{l}{\textbf{Proc} MaxSubSum(a, left, right)\textcolor{blue}{// a为整数序列}} \\
 & \multicolumn{4}{l}{\textbf{if} left $<$ right \textbf{then}} \\
 &  & \multicolumn{3}{l}{middle = $\lfloor\textrm{(left+right)}/2\rfloor$} \\
 &  & \multicolumn{3}{l}{MaxSumLeft = MaxSubSum(a, left, middle)} \\
 &  & \multicolumn{3}{l}{MaxSumRight = MaxSubSum(a,  middle+1, right)} \\
 &  & \multicolumn{3}{l}{sumleft =0; maxleft = 0;} \\
 &  & \multicolumn{3}{l}{\textbf{for} i \textbf{from} middle \textbf{to} left} \\
 &  &  & \multicolumn{2}{l}{sumleft = sumleft + a(i)} \\
 &  &  & \multicolumn{2}{l}{maxleft = \textbf{max}(sumleft, maxleft)} \\
 &  & \multicolumn{3}{l}{\textbf{end\{for\}}} \\
 &  & \multicolumn{3}{l}{sumright =0; maxright = 0;} \\
 &  & \multicolumn{3}{l}{\textbf{for} i \textbf{from} middle+1 \textbf{to} right} \\
 &  &  & \multicolumn{2}{l}{sumright = sumright + a(i)} \\
 &  &  & \multicolumn{2}{l}{maxright = \textbf{max}(sumright, maxright)} \\
 &  & \multicolumn{3}{l}{\textbf{end\{for\}}} \\
 &  & \multicolumn{3}{l}{SumCombine = maxleft+maxright}\\
 &  & \multicolumn{3}{l}{MaxLightRight = \textbf{max}(MaxSumLeft, MaxSumRight)} \\
 &  & \multicolumn{3}{l}{MaxSum = \textbf{max}(SumCombine, MaxLightRight)} \\
 &  & \multicolumn{3}{l}{\textbf{return} MaxSum} \\
 & \multicolumn{4}{l}{\textbf{else}} \\
 &  & \multicolumn{3}{l}{MaxSum = \textbf{max}(a(left),0)} \\
 &  & \multicolumn{3}{l}{\textbf{return} MaxSum} \\
\multicolumn{5}{l}{\textbf{end\{Proc\}}} \\
\hline
\end{tabular}
\end{table}\\
该算法的时间复杂度$T(n)$有递归关系式$T(n)=2T(n/2)+O(n), n>1$, $T(1) = O(1)$. 满足典型的分治算法递归关系式, 故$T(n) = O(n\log n)$
\item 设$S(j)=\max\{a_1+a_2+\cdots + a_j, a_2+a_3+\cdots + a_j, \cdots, a_{j-1}+a_j,a_j\} = \max\limits_{1\leq i\leq j}\{\sum\limits_{k=i}\limits^ja_i\}$. 则子段和的最大值为$\max\{S(1), S(2), \cdots, S(n)\}$. 由S(j)的定义得
    \[
    S(j) = \max\limits_{1\leq i\leq j-1}\{a_j+\sum\limits_{k=i}\limits^{j-1}a_i, a_j\} = \max\{a_j+S(j-1), a_j\}
    \]
因此, $S(j)$具有最优子结构. 其动态规划最优子结构公式为:
\[
S(j) = \max\{S(j-1)+a_j, a_j\},{~} 1\leq j\leq n
\]
根据以上分析, 可得到解最大子段动态规划算法的伪代码, 见表\ref{Maxsum}.
该算法的主要计算量取决于程序对$i$的for盾环, 循环体内的计算量为$O(1)$. 因此该算法的时间复杂度为$O(n)$.
\begin{table}[!htb]
\centering
\caption{\label{Maxsum}解最大子段动态规划算法的伪代码}
\begin{tabular}{llll}
\hline
\multicolumn{4}{l}{\textbf{Proc} MaxSum(a)\textcolor{blue}{//a为整数序列}} \\
 & \multicolumn{3}{l}{n = $|a|$} \\
 & \multicolumn{3}{l}{maxsum = 0; sum = 0;} \\
 & \multicolumn{3}{l}{I = 0; J = 0; \textcolor{blue}{//初始化I,J. I,J 分别为最大子段的始终元素下标}} \\
 & \multicolumn{3}{l}{\textbf{for} i \textbf{from} 1 \textbf{to} n} \\
 &  & \multicolumn{2}{l}{\textbf{if} sum $> 0$ \textbf{then}} \\
 &  &  & sum = sum + a[i]; \\
 &  & \multicolumn{2}{l}{\textbf{else}} \\
 &  &  & sum = a[i]; \\
 &  &  & I = i; \\
 &  & \multicolumn{2}{l}{\textbf{end\{if\}}} \\
 &  & \multicolumn{2}{l}{\textbf{if} sum $>$ maxsum \textbf{then}} \\
 &  &  & maxsum = sum; \\
 &  &  & J = i; \\
 &  & \multicolumn{2}{l}{\textbf{end\{if\}}} \\
 & \multicolumn{3}{l}{\textbf{end\{for\}}} \\
 & \multicolumn{3}{l}{\textbf{return} maxsum, I, J;} \\
\multicolumn{4}{l}{\textbf{end\{Proc\}}} \\
\hline
\end{tabular}
\end{table}
\end{enumerate}

\end{solution}
